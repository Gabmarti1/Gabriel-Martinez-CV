%%%%%%%%%%%%%%%%%
% This is an example CV created using altacv.cls (v1.1.5, 1 December 2018) written by
% LianTze Lim (liantze@gmail.com), based on the
% Cv created by BusinessInsider at http://www.businessinsider.my/a-sample-resume-for-marissa-mayer-2016-7/?r=US&IR=T
%
%% It may be distributed and/or modified under the
%% conditions of the LaTeX Project Public License, either version 1.3
%% of this license or (at your option) any later version.
%% The latest version of this license is in
%%    http://www.latex-project.org/lppl.txt
%% and version 1.3 or later is part of all distributions of LaTeX
%% version 2003/12/01 or later.
%%%%%%%%%%%%%%%%

%% If you are using \orcid or academicons
%% icons, make sure you have the academicons
%% option here, and compile with XeLaTeX
%% or LuaLaTeX.
% \documentclass[10pt,a4paper,academicons]{altacv}

%% Use the "normalphoto" option if you want a normal photo instead of cropped to a circle
% \documentclass[10pt,a4paper,normalphoto]{altacv}

\documentclass[10pt,a4paper,ragged2e]{altacv}

%% AltaCV uses the fontawesome and academicon fonts
%% and packages.
%% See texdoc.net/pkg/fontawecome and http://texdoc.net/pkg/academicons for full list of symbols. You MUST compile with XeLaTeX or LuaLaTeX if you want to use academicons.

% Change the page layout if you need to
\geometry{left=1cm,right=9cm,marginparwidth=6.8cm,marginparsep=1.2cm,top=1.25cm,bottom=1.25cm}

% Change the font if you want to, depending on whether
% you're using pdflatex or xelatex/lualatex
\ifxetexorluatex
  % If using xelatex or lualatex:
  \setmainfont{Carlito}
\else
  % If using pdflatex:
  \usepackage[utf8]{inputenc}
  \usepackage[T1]{fontenc}
  \usepackage[default]{lato}
\fi
\usepackage{graphicx}
% Change the colours if you want to
\definecolor{DarkBlue}{HTML}{00008B}
\definecolor{VividPurple}{HTML}{3E0097}
\definecolor{SlateGrey}{HTML}{2E2E2E}
\definecolor{LightGrey}{HTML}{666666}

\colorlet{heading}{DarkBlue}
\colorlet{accent}{DarkBlue}
\colorlet{emphasis}{SlateGrey}
\colorlet{body}{LightGrey}

% Change the bullets for itemize and rating marker
% for \cvskill if you want to
\renewcommand{\itemmarker}{{\small\textbullet}}
\renewcommand{\ratingmarker}{\faCircle}

%% sample.bib contains your publications
\addbibresource{sample.bib}

\begin{document}
\name{\Huge Gabriel Martinez Tirado}
\tagline{\LARGE Bioinformatics and Data science}

\personalinfo{%
  % Not all of these are required!
  % You can add your own with \printinfo{symbol}{detail}
  \email{gabriel.martinez@uni.lu}
%   \phone{000-00-0000}
 %  \mailaddress{Adelaida Muro 1E, Bajo B}
  \location{Luxembourg}
  \linkedin{www.linkedin.com/in/gabriel-martínez-tirado-903436231}
  \github{github.com/Gabmarti1} % I'm just making this up though.
%   \orcid{orcid.org/0000-0000-0000-0000} % Obviously making this up too. If you want to use this field (and also other academicons symbols), add "academicons" option to \documentclass{altacv}
}

%% Make the header extend all the way to the right, if you want.
\begin{fullwidth}
\makecvheader
\end{fullwidth}

%% Depending on your tastes, you may want to make fonts of itemize environments slightly smaller
\AtBeginEnvironment{itemize}{\small}

%% Provide the file name containing the sidebar contents as an optional parameter to \cvsection.
%% You can always just use \marginpar{...} if you do
%% not need to align the top of the contents to any
%% \cvsection title in the "main" bar.
\cvsection[page1sidebar]{Experience}

\cvevent{PhD in translational data science, Digital Medicine}{University of Luxembourg}{September 2022 -- currently}{Luxembourg}
\begin{itemize}
\item  Machine Learning and deep learning algorithms
\item  Statistics analysis and data visualization
\item  Databases manage with SQL
\item Communication and collaboration with clinical team
\item Knowledge of Linux, command line and bash scripting
\end{itemize}

\divider

\cvevent{Bioinformatician and ML developer}{UPM physics department}{January 2022 -- july 2022}{Madrid}
\begin{itemize}
\item Developing and optimization of ML models in the field of preventive medicine
\item Communication and collaboration with clinical team
\end{itemize}

\divider

\cvevent{NGS data analyst}{CSIC}{January 2021 --  July 2021}{Valencia}
\begin{itemize}
\item RNA-seq analysis, for differential expresion analysis and functional annotation.
\item Used of multiple open-source programs in linux such as RNAstar,bowtie,deseq2, trinity, hisat2, and conda environments.
\item Linux environment and command line, Python and Bash programming.

\divider
\end{itemize}

\cvevent{Bioinformatics intership}{CSIC}{may 2020 -- december 2020}{Valencia}

\begin{itemize}
\item Python
\item Data visualization
\item Bash scripting
\item Linux environments
\end{itemize}
\divider
% \cvevent{Product Engineer}{Google}{23 June 1999 -- 2001}{Palo Alto, CA}

% \begin{itemize}
% \item Joined the company as employe \#20 and female employee \#1
% \item Developed targeted advertisement in order to use user's search queries and show them related ads
% \end{itemize}

\cvsection{Professional interests}

% Adapted from @Jake's answer from http://tex.stackexchange.com/a/82729/226
% \wheelchart{outer radius}{inner radius}{
% comma-separated list of value/text width/color/detail}
% Some ad-hoc tweaking to adjust the labels so that they don't overlap
\wheelchart{1.5cm}{0.5cm}{%
  10/10em/accent!30/Data scientist,
  25/9em/accent!60/Deep learning,
  5/13em/accent!10/\footnotesize\\[1ex]NGS analysis,
  20/15em/accent!40/Genomics,
  5/8em/accent!20/\footnotesize OOP,
  30/9em/accent/Machine Learning,
  5/8em/accent!20/Statistics
}
\end{document}
